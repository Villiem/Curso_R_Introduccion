\documentclass[]{article}
\usepackage{lmodern}
\usepackage{amssymb,amsmath}
\usepackage{ifxetex,ifluatex}
\usepackage{fixltx2e} % provides \textsubscript
\ifnum 0\ifxetex 1\fi\ifluatex 1\fi=0 % if pdftex
  \usepackage[T1]{fontenc}
  \usepackage[utf8]{inputenc}
\else % if luatex or xelatex
  \ifxetex
    \usepackage{mathspec}
  \else
    \usepackage{fontspec}
  \fi
  \defaultfontfeatures{Ligatures=TeX,Scale=MatchLowercase}
\fi
% use upquote if available, for straight quotes in verbatim environments
\IfFileExists{upquote.sty}{\usepackage{upquote}}{}
% use microtype if available
\IfFileExists{microtype.sty}{%
\usepackage{microtype}
\UseMicrotypeSet[protrusion]{basicmath} % disable protrusion for tt fonts
}{}
\usepackage[margin=1in]{geometry}
\usepackage{hyperref}
\hypersetup{unicode=true,
            pdftitle={Curso introductorio de R},
            pdfauthor={Manuel Toral},
            pdfborder={0 0 0},
            breaklinks=true}
\urlstyle{same}  % don't use monospace font for urls
\usepackage{graphicx,grffile}
\makeatletter
\def\maxwidth{\ifdim\Gin@nat@width>\linewidth\linewidth\else\Gin@nat@width\fi}
\def\maxheight{\ifdim\Gin@nat@height>\textheight\textheight\else\Gin@nat@height\fi}
\makeatother
% Scale images if necessary, so that they will not overflow the page
% margins by default, and it is still possible to overwrite the defaults
% using explicit options in \includegraphics[width, height, ...]{}
\setkeys{Gin}{width=\maxwidth,height=\maxheight,keepaspectratio}
\IfFileExists{parskip.sty}{%
\usepackage{parskip}
}{% else
\setlength{\parindent}{0pt}
\setlength{\parskip}{6pt plus 2pt minus 1pt}
}
\setlength{\emergencystretch}{3em}  % prevent overfull lines
\providecommand{\tightlist}{%
  \setlength{\itemsep}{0pt}\setlength{\parskip}{0pt}}
\setcounter{secnumdepth}{0}
% Redefines (sub)paragraphs to behave more like sections
\ifx\paragraph\undefined\else
\let\oldparagraph\paragraph
\renewcommand{\paragraph}[1]{\oldparagraph{#1}\mbox{}}
\fi
\ifx\subparagraph\undefined\else
\let\oldsubparagraph\subparagraph
\renewcommand{\subparagraph}[1]{\oldsubparagraph{#1}\mbox{}}
\fi

%%% Use protect on footnotes to avoid problems with footnotes in titles
\let\rmarkdownfootnote\footnote%
\def\footnote{\protect\rmarkdownfootnote}

%%% Change title format to be more compact
\usepackage{titling}

% Create subtitle command for use in maketitle
\newcommand{\subtitle}[1]{
  \posttitle{
    \begin{center}\large#1\end{center}
    }
}

\setlength{\droptitle}{-2em}

  \title{Curso introductorio de R}
    \pretitle{\vspace{\droptitle}\centering\huge}
  \posttitle{\par}
    \author{Manuel Toral}
    \preauthor{\centering\large\emph}
  \postauthor{\par}
      \predate{\centering\large\emph}
  \postdate{\par}
    \date{Octubre 2018}


\begin{document}
\maketitle

\section{¿Dónde es el curso?}\label{donde-es-el-curso}

El curso se llevará a cabo en las instalaciones \emph{Cubic Idea} en
Colima 158, Colonia Roma Norte, en la Ciudad de México.

\section{¿Cómo contactar al
instructor?}\label{como-contactar-al-instructor}

\begin{itemize}
\tightlist
\item
  Correo electrónico:
  \href{mailto:jmtoralcruz@gmail.com}{\nolinkurl{jmtoralcruz@gmail.com}}
\item
  Teléfono: Será provisto el primer día del curso.
\end{itemize}

\section{Objetivo del curso}\label{objetivo-del-curso}

El objetivo del curso es que aprendas a utilizar efectivamente
\texttt{R} para su vida profesional diaria. Así pues, al final del curso
se espera desarrolles habilidades con cuatro objetivos particulares.

\begin{enumerate}
\def\labelenumi{\arabic{enumi}.}
\item
  \emph{Ahorrar tiempo en el procesamiento de información:} El talento
  de las personas no debe perderse en procesar y talachear bases de
  datos. \texttt{R} es una herramienta útil para hacer esta tarea de
  manera más rápida que con una hoja de cálculo.
\item
  \emph{Pensar de manera estructurada:} Pensar nuestros procesos de
  análisis desde una perspectiva estructurada nos da la oportunidad de
  encontrar respuestas a nuestras preguntas de manera más sencilla.
\item
  \emph{Aumentar los límites de acceso a la información:} Muchas de las
  bases de datos que se usan (y se necesitan) hoy en día están en
  formatos como \texttt{JSON} o \texttt{SHP}. \texttt{R} es capaz de
  leerlos y tranformarlos en información útil.
\item
  \emph{Automatizar tareas:} Las computadoras fueron inventadas para
  repetir procesos. Una y otra vez. Cada vez que un ser humano repite
  una actividad frente a una computadora, está haciendo algo mal. La
  idea es evitar ese tipo de repeticiones e instruir a la computadora a
  que las haga.
\item
  \emph{Aprovechar a la comunidad:} Las computadoras fueron inventadas
  para repetir procesos. Una y otra vez. Cada vez que un ser humano
  repite una actividad frente a una computadora, está haciendo algo mal.
  La idea es evitar ese tipo de repeticiones e instruir a la computadora
  a que las haga.
\end{enumerate}

\section{Material guía}\label{material-guia}

\subsection{Cheatsheets}\label{cheatsheets}

\begin{itemize}
\item
  \href{https://www.rstudio.com/wp-content/uploads/2016/10/r-cheat-sheet-3.pdf}{\textbf{Base
  R}}
\item
  \href{https://github.com/rstudio/cheatsheets/raw/master/lubridate.pdf}{\textbf{Variables
  de tiempo}}
\item
  \href{https://github.com/rstudio/cheatsheets/raw/master/data-transformation.pdf}{\textbf{Transformación
  de datos con \texttt{dplyr}}}
\item
  \href{https://www.rstudio.com/wp-content/uploads/2015/03/rmarkdown-reference.pdf}{\textbf{RMarkdown}}
\item
  \href{https://github.com/rstudio/cheatsheets/raw/master/data-visualization-2.1.pdf}{\textbf{ggplot2}}
\end{itemize}

\subsection{Libros}\label{libros}

(Da click en el nombre del libro para abrirlo.)

\begin{itemize}
\tightlist
\item
  \href{https://bookdown.org/ndphillips/YaRrr/}{La guía pirata de R -
  Phillips}
\item
  \href{http://r4ds.had.co.nz/}{R for Data Science - Grolemund \&
  Wickham}
\item
  \href{https://geocompr.robinlovelace.net/}{Geocomputation with R -
  Lovelace, Nowosad \& Muenchow}
\end{itemize}

\subsection{Páginas webs}\label{paginas-webs}

\begin{itemize}
\tightlist
\item
  \href{https://stackoverflow.com/}{Stack Overflow}
\item
  \href{http://www.sthda.com/english/}{STHDA}
\end{itemize}

\section{Requisitos}\label{requisitos}

\begin{itemize}
\item
  Es conveniente que elijas un problema que quieras resolver en tu día a
  día y poder hacer ejercicios con información de ese tipo. Te
  recomiendo que me mandes un día antes alguna base con la que quieras
  trabajar o algún tema que quieras explorar con datos.
\item
  El único requisito es tener instaladas las tres herramientas
  necesarias. Aquí unas breves instrucciones. Si necesitas ver el paso a
  paso, consulta el archivo \texttt{Instalar\_Paso\_a\_Paso.pdf} en el
  repositorio.
\item
  Haber instalado \texttt{R} en la computadora.

  \begin{enumerate}
  \def\labelenumi{\arabic{enumi}.}
  \tightlist
  \item
    Da clic \href{https://cran.itam.mx/}{\textbf{aquí}}.
  \item
    Seleccionar el archivo según tu sistema operativo.
  \item
    Seleccionar el link de \textbf{install R for the first time.}
  \item
    Descargar el archivo y abrirlo para seguir las insturcciones.
  \end{enumerate}
\item
  Haber instalado \emph{RStudio} en la computadora.

  \begin{enumerate}
  \def\labelenumi{\arabic{enumi}.}
  \tightlist
  \item
    Da clic
    \href{https://www.rstudio.com/products/rstudio/download/\#download}{\textbf{aquí}}.
  \item
    Descarga el instalador de la sección \textbf{Installers for
    Supported Platforms} de acuerdo a tu sistema operativo.
  \item
    Da clik en el archivo descargado y sigue las instrucciones.
  \end{enumerate}
\item
  Haber instalado \LaTeX. El link está aquí.

  \begin{enumerate}
  \def\labelenumi{\arabic{enumi}.}
  \tightlist
  \item
    Este está más complicado.
    \href{https://www.latex-project.org/get/}{\textbf{Aquí}} están las
    instrucciones, pero sí está muy engorroso, en la clase lo
    resolvemos.
  \end{enumerate}
\item
  \emph{Estar preparadx para la frustración}.
\end{itemize}

\section{Uso profesional de R y
RStudio}\label{uso-profesional-de-r-y-rstudio}

Gran parte de las cosas que uno hace en \texttt{R} forman parte de
reportes, presentaciones y textos. En esta primera sección aprenderemos
cómo usar \texttt{R} en \emph{RStudio} de manera profesional a través de
\emph{R Markdown}. Los temas son los siguientes:

\begin{enumerate}
\def\labelenumi{\arabic{enumi}.}
\tightlist
\item
  Crear archivos en \emph{R Markdown} y ``knitearlos'' para generar
  archivos estéticamente atractivos y claros.
\item
  Entender el uso de los \emph{chunks}.
\item
  Entender \emph{RStudio} y sus componentes.
\item
  Uso de la consola y la terminal.
\item
  Uso de comandos \texttt{echo}, \texttt{warning}, \texttt{fig.*} y
  \texttt{eval} en la escritura de \emph{R Markdown}.
\item
  Exportar \emph{PDF}s, archivos en \emph{HTML} y presentaciones en
  \emph{Beamer}.
\item
  ¿Cómo instalar librerías?
\end{enumerate}

\section{\texorpdfstring{Brevísima introducción a
\texttt{R\ base}}{Brevísima introducción a R base}}\label{brevisima-introduccion-a-r-base}

En \texttt{R} hay
\href{http://www.science.smith.edu/~amcnamara/Syntax-cheatsheet.pdf}{tres
sintaxis}: \emph{Signo de dólar}, \emph{Formula} y \emph{La sintaxis del
tidyverso}. Mi favorita y, a mi parecer, la más útil y fácil es la del
tidyverso. No obstante, repasaremos algunas funciones de la sintaxis
básica del lenguaje para que no te agarren en curva.

\begin{enumerate}
\def\labelenumi{\arabic{enumi}.}
\tightlist
\item
  Como crear objetos (y por qué este método de trabajo es superior a
  hojas de cálculo y ventanas de software)
\item
  Tipos de datos (vectores, listas, data frames, matrices, strings)
\item
  El método de la \emph{indexación} y cómo usar el famoso \texttt{\$}.
\item
  Manipulaciones básicas para explorar bases.
\item
  Funciones básicas para obtener estadística descriptiva.
\item
  Uso de variables de timempo.
\end{enumerate}

\section{\texorpdfstring{El
\texttt{tidyverse}}{El tidyverse}}\label{el-tidyverse}

\begin{enumerate}
\def\labelenumi{\arabic{enumi}.}
\tightlist
\item
  ¿Qué es y cómo usar un pipe \texttt{\%\textgreater{}\%}?
\item
  Filtrar y filtrar con condiciones
\item
  Seleccionar variables y modificar sus nombres
\item
  Covertir de \emph{long} a \emph{wide} y viceversa.
\item
  Agrupar, convertir y ``colapsar'' bases de datos
\item
  Uso de \texttt{grep} y una muy breve introducción a \emph{regular
  expressions}.
\end{enumerate}

\section{Importar (y usar efectivamente) bases de
datos}\label{importar-y-usar-efectivamente-bases-de-datos}

\begin{enumerate}
\def\labelenumi{\arabic{enumi}.}
\tightlist
\item
  Importar datos en formatos \texttt{.xls} (\emph{MS Excel}),
  \texttt{.dta} (\emph{SATA}), \texttt{.sav} (\emph{SPSS}),
  \texttt{.dta} (\emph{SATA}), \texttt{.txt} y \texttt{.csv}.
\item
  Funciones básicas para el análisis estadístico (\texttt{lm},
  \texttt{predict}, \texttt{stepAIC}, \texttt{anova}).
\item
  Manipulación se datos: hacer muestras y subsets, unir bases con
  \texttt{merge}, creación de variables binomiales, entre otras.
\item
  Manipulación de \emph{strings}
\item
  Introducción a la visualización de datos con \texttt{ggplot}.
\end{enumerate}

\section{Visualización}\label{visualizacion}

\begin{enumerate}
\def\labelenumi{\arabic{enumi}.}
\tightlist
\item
  Introducción a al gramática de \texttt{ggplot2} y al uso de
  \texttt{aes()}.
\item
  Introducción a la lógica de capas.
\item
  Elementos estructurales de las visualizaciones.
\item
  Modificación de elementos guía: títulos, subtítulos, títulos de ejes,
  thiks, escalas, límites, grids y guías.
\item
  \texttt{qplot}, esa salida fácil.
\item
  Tipos de visualización con respecto a la naturaleza de los datos.
\end{enumerate}

\section{Últimas fronteras}\label{ultimas-fronteras}

\begin{enumerate}
\def\labelenumi{\arabic{enumi}.}
\tightlist
\item
  Breve introducción a la minería de texto.
\item
  Breve introducción a los datos geográficos.
\item
  Breve introducción al scrapeo de información.
\end{enumerate}


\end{document}
